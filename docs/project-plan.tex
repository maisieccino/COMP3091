% arara: xelatex: { synctex: true, action: nonstopmode, options: "-halt-on-error" }
% arara: clean: { files: [ mattbell-coursework1.aux, mattbell-coursework1.bbl, mattbell-coursework1.bcf, mattbell-coursework1.blg, mattbell-coursework1.log, mattbell-coursework1.out, mattbell-coursework1.synctex.gz, mattbell-coursework1.run.xml ] }
\documentclass{article}

\usepackage[margin=1in]{geometry} \pagestyle{headings} \usepackage{fontspec}
\usepackage{titlesec} \usepackage{titling} \usepackage{xcolor}

\definecolor{colorDoc}{HTML}{333333}

\newfontfamily\headingfont[]{Circular Std Book}
\setmonofont[Color=colorDoc]{Operator Mono}
\setmainfont[Color=colorDoc]{Roboto}
\titleformat*{\section}{\LARGE\bfseries\headingfont}
\titleformat*{\subsection}{\Large\bfseries\headingfont}
\titleformat*{\subsubsection}{\large\bfseries\headingfont}
\renewcommand{\maketitlehooka}{\headingfont}

\title{\protect\parbox{\textwidth}{\protect\centering %
Tracking Wildlife Counts Using the Internet Of Things: \\
Project Plan \\
\large Supervisor: Dr Kevin Bryson}} \author{Matthew Bell}

\begin{document} \maketitle

\section*{Aims \& Objectives}
\begin{itemize}
    \item To build an Internet of Things solution
        that can accurately detect the type and presence of various animals
    \item To implement a computer vision solution that allows the detection of
        animals onboard a low powered, portable computer.
    \item Learn a deeper understanding of the Internet of Things and
        sensor-driven systems through working on a technically-rigourous project
\end{itemize}

\section*{Expected Outcome \& Deliverables}
\begin{itemize}
    \item A program built for the Creator Ci40 prototyping board that can read
        in an image sent by the external camera module, and classify the types
        and counts of animals in the image, before sending this data to a base
        station via a LORA connection
    \item A program for the IR clicker device to detect movement and send a
        message to a nearby camera device
    \item A program for the camera device to take a photo when it receives a
        command via 6LoWPAN and send it to the Ci40 board
    \item A simple cloud service to store and display results received from the
        prototype boards
    \item A rigorous testing regime for as much code as possible
    \item (If possible) a real-life test of the system in an uncontrolled
        environment (i.e. a park or zoo)
    \item A design specification for the programs.
\end{itemize}

\section*{Work Plan}
\begin{itemize}
    \item Project start to end of October, search for exisitng solutions and
        formalise requiremnets with supervisor
    \item October to November, begin to get familiar with the prototyping boards
        and their associated toolings
    \item November to end of February, work on getting the following systems
        operational:
        \begin{itemize}
            \item Triggering camera shutter via command sent over 6LoWPAN
            \item Detecting movement from the IR sensing device and sending a
                6LoWPAN command
            \item Deciphering data sent from camera to Ci40 into a processible
                image
            \item Work on image recognition of images on the Ci40.
        \end{itemize}
    \item Mid-February to end of March, write project report and build simple
        cloud system.
\end{itemize}

\end{document}
