% arara: xelatex: { synctex: true, action: nonstopmode, options: "-halt-on-error" }
% arara: clean: { files: [ interim-report.aux, interim-report.bbl, interim-report.bcf, interim-report.blg, interim-report.log, interim-report.out, interim-report.synctex.gz, interim-report.run.xml ] }
\documentclass{article}

\usepackage[margin=1in]{geometry}
\pagestyle{headings}
\usepackage{fontspec}
\usepackage{titlesec}
\usepackage{titling}
\usepackage{xcolor}

\definecolor{colorDoc}{HTML}{333333}

\newfontfamily\headingfont[]{Circular Std Book}
\setmonofont[Color=colorDoc]{Operator Mono}
\setmainfont[Color=colorDoc]{Roboto}
\titleformat*{\section}{\LARGE\bfseries\headingfont}
\titleformat*{\subsection}{\Large\bfseries\headingfont}
\titleformat*{\subsubsection}{\large\bfseries\headingfont}
\renewcommand{\maketitlehooka}{\headingfont}

\title{\protect\parbox{\textwidth}{\protect\centering %
Tracking Wildlife Counts Using the Internet Of Things: \\
Project Plan \\ \vspace{1em}
\large Supervisor: Dr Kevin Bryson}} \author{Matthew Bell}

\begin{document} \maketitle

\begin{center}
    \huge \bfseries \headingfont Interim Report
\end{center}

\section*{Current Progress}
\begin{itemize}
    \item Designed architecture for project. The solution will consist of two
    6LoWPAN "clicker" devices connected to a central base station. The base
    station will use a python program to control the connections to the
    clicker devices (receiving IR activation signal and camera information),
    and will then take the image received from the camera and classify the
    its content.
    \item Set up project toolchain. This took far longer than expected,
    because of the intricacies of the devices being used in the project. The
    base station has had a lot of trouble connecting to the Internet so
    development has been slow in that regard, and until very recently I had
    trouble programming the clicker devices. However, these issues have been
    fixed, and despite being a little behind on progress, I can now proceed
    to program these devices.
    \item Learning about convoluted neural networks and computer vision. The
    above roadblocks led me to start working on the image classification part
    of the project earlier than planned, and quickly decided upon using the
    Tensorflow library as the main way of classifying images from the camera.
    In the short term, I'm using Google's Inception model, which boasts a
    high accuracy rating on a generic image classification task (ImageNet).
    By the end of the project, if I have time, I'd still like to create my
    own classifier that is bespoke for this project, and trained on similar
    test data.
    \item Identified similar projects and how they approached their problems.
    During my research, I found the Snapshot Serengeti project and how they
    were using human volunteers to help classify their images and to train
    their own model for classifying images. This wasn't especially helpful on
    its own, however it did mean that I discovered a high quality training
    set of over three thousand labelled images, which classified 40 different
    types of animal, which I can use to train my own model.
\end{itemize}

\section*{Outstanding Tasks}

\begin{itemize}
    \item Program the camera and Infrared sensor boards. This should take
    less time now that I've passed one of the main roadblocks (setting up the
    tooling), however completing this is the highest priority. There might be
    some further issues with developing the camera shutter program, since
    prior work by other students has proven unsuccessful, so this is where
    I'll be dedicating a large amount of my time.
    \item Programming the base station's programs. I need to write one single
    python program that will communicate with the clicker boards over
    6LoWPAN, receive camera data when the shutter is activated, process and
    classify the contents of the image, and then send those results over
    Lorawan.
    \item Creating a simple web app for receiving and processing results.
    This is very low priority and will probably take the least amount of time
    to build. This will be completed as one of the final tasks of the
    project.
\end{itemize}

\end{document}
