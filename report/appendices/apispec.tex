\chapter{API Specification}

\section{Base Station}

Endpoint: \texttt{/basestation/:id}

\noindent
This endpoint concerns the base stations registered on the system.\\

\noindent
\textbf{Optional Parameters}
\begin{itemize}
   \item \texttt{:id} \textemdash The id of the selected base station (not required for getting list of all base stations or adding new station).
\end{itemize}


\subsection{Get a list of all base stations}

\noindent
\textbf{Method}: \texttt{GET}

\noindent
\textbf{Request Format}: \texttt{<n/a>}

\noindent
\textbf{Response Content-Type}: \texttt{JSON}

\noindent
\textbf{Expected Status Code}: \texttt{200 OK}

\noindent
\textbf{Example Response}:
\begin{minted}{json}
{ 
  "error": "",
  "content": [
    {
      "id": "00000000-0000-0000-0000-000000000000",
      "name": "Test Station",
      "lat": 0,
      "lng": 0
    }
  ]
}
\end{minted}


\subsection{Add a new base station}

\noindent
\textbf{Method}: \texttt{POST}

\noindent
\textbf{Request Format}: \texttt{JSON}
\begin{minted}{json}
{ 
  "name": "Test Station",
  "lat": 0,
  "lng": 0,
}
\end{minted}

\noindent
\textbf{Required Fields}: \texttt{<none>}

\noindent
\textbf{Response Content-Type}: \texttt{JSON}

\noindent
\textbf{Expected Status Code}: \texttt{201 Created}

\noindent
\textbf{Example Response}:
\begin{minted}{json}
{ 
  "error": "",
  "content": {
    "id": "00000000-0000-0000-0000-000000000000",
    "name": "Test Station",
    "lat": 0,
    "lng": 0
  }
}
\end{minted}


\subsection{Modify an existing base station}

\noindent
\textbf{Method}: \texttt{PATCH}

\noindent
\textbf{Request Format}: \texttt{JSON}
\begin{minted}{json}
{ 
  "name": "Test Station",
  "lat": 0,
  "lng": 0,
}
\end{minted}

\noindent
\textbf{Required Fields}: \texttt{<none>}

\noindent
\textbf{Response Content-Type}: \texttt{JSON}

\noindent
\textbf{Expected Status Code}: \texttt{200 OK}

\noindent
\textbf{Example Response}:
\begin{minted}{json}
{ 
  "error": "",
  "content": {
    "id": "00000000-0000-0000-0000-000000000000",
    "name": "Test Station",
    "lat": 0,
    "lng": 0
  }
}
\end{minted}


\subsection{Delete an existing base station}

\noindent
\textbf{Method}: \texttt{DELETE}

\noindent
\textbf{Request Format}: \texttt{<n/a>}

\noindent
\textbf{Required Fields}: \texttt{<none>}

\noindent
\textbf{Response Content-Type}: \texttt{<none>}

\noindent
\textbf{Expected Status Code}: \texttt{204 No Content}



\section{Base Station Sensor Pairs}

Endpoint: \texttt{/basestation/:id/sensorpairs}

\noindent
This endpoint concerns the sensor pairs connected to a base station.\\


\subsection{Get a list of a base station's sensor pairs}

\noindent
\textbf{Method}: \texttt{GET}

\noindent
\textbf{Request Format}: \texttt{<n/a>}

\noindent
\textbf{Response Content-Type}: \texttt{JSON}

\noindent
\textbf{Expected Status Code}: \texttt{200 OK}

\noindent
\textbf{Example Response}:
\begin{minted}{json}
{ 
  "error": "",
  "content": [
    {
      "id": "00000000-0000-0000-0000-000000000000",
      "name": "Treetop Sensor",
      "lat": 0,
      "lng": 0,
      "camera_id": 234,
      "motion_id": 120
    }
  ]
}
\end{minted}


\subsection{Add a new sensor pair to a base station}

\noindent
\textbf{Method}: \texttt{POST}

\noindent
\textbf{Request Format}: \texttt{JSON}
\begin{minted}{json}
{ 
  "name": "Treetop Sensor",
  "lat": 0,
  "lng": 0,
  "camera_id": 123,
  "motion_id": 456
}
\end{minted}

\noindent
\textbf{Required Fields}: \texttt{camera\_id, motion\_id}

\noindent
\textbf{Response Content-Type}: \texttt{JSON}

\noindent
\textbf{Expected Status Code}: \texttt{201 Created}

\noindent
\textbf{Example Response}:
\begin{minted}{json}
{ 
  "error": "",
  "content": {
    "id": "00000000-0000-0000-0000-000000000000",
    "name": "Treetop Sensor",
    "lat": 0,
    "lng": 0,
    "camera_id": 123,
    "motion_id": 456
  }
}
\end{minted}



\section{Sensor Pair}

Endpoint: \texttt{/sensorpair/:id}

\noindent
This endpoint concerns all sensor pairs registered on the system.\\

\noindent
\textbf{Optional Parameters}
\begin{itemize}
   \item \texttt{:id} \textemdash The id of the selected sensor pair (not required for getting list of all sensor pairs or adding new pair).
\end{itemize}


\subsection{Get a list of all sensor pairs}

\noindent
\textbf{Method}: \texttt{GET}

\noindent
\textbf{Request Format}: \texttt{<n/a>}

\noindent
\textbf{Response Content-Type}: \texttt{JSON}

\noindent
\textbf{Expected Status Code}: \texttt{200 OK}

\noindent
\textbf{Example Response}:
\begin{minted}{json}
{ 
  "error": "",
  "content": [
    {
      "id": "00000000-0000-0000-0000-000000000000",
      "name": "Treetop Sensor",
      "lat": 0,
      "lng": 0,
      "camera_id": 123,
      "motion_id": 456
    }
  ]
}
\end{minted}


\subsection{Modify an existing sensor pair}

\noindent
\textbf{Method}: \texttt{PATCH}

\noindent
\textbf{Request Format}: \texttt{JSON}
\begin{minted}{json}
{ 
  "name": "Treetop Sensor",
  "lat": 0,
  "lng": 0,
}
\end{minted}

\noindent
\textbf{Required Fields}: \texttt{<none>}

\noindent
\textbf{Response Content-Type}: \texttt{JSON}

\noindent
\textbf{Expected Status Code}: \texttt{200 OK}

\noindent
\textbf{Example Response}:
\begin{minted}{json}
{ 
  "error": "",
  "content": {
    "id": "00000000-0000-0000-0000-000000000000",
    "name": "Treetop Sensor",
    "lat": 0,
    "lng": 0,
    "camera_id": 123,
    "motion_id": 456
  }
}
\end{minted}


\subsection{Delete an existing sensor pair}

\noindent
\textbf{Method}: \texttt{DELETE}

\noindent
\textbf{Request Format}: \texttt{<n/a>}

\noindent
\textbf{Required Fields}: \texttt{<none>}

\noindent
\textbf{Response Content-Type}: \texttt{<none>}

\noindent
\textbf{Expected Status Code}: \texttt{204 No Content}



\section{Sensor Pair Reading}

Endpoint: \texttt{/sensorpair/:pairid/reading/:readingid}

\noindent
This endpoint concerns the base stations registered on the system.\\

\noindent
\textbf{Required Parameters}
\begin{itemize}
   \item \texttt{:pairid} \textemdash The id of the selected sensor pair.
\end{itemize}

\noindent
\textbf{Optional Parameters}
\begin{itemize}
   \item \texttt{:readingid} \textemdash The id of the selected reading (not
   required for getting list of all readings or adding new reading).
\end{itemize}


\subsection{Get a list of all readings for a sensor}

\noindent
\textbf{Method}: \texttt{GET}

\noindent
\textbf{Request Format}: \texttt{<n/a>}

\noindent
\textbf{Response Content-Type}: \texttt{JSON}

\noindent
\textbf{Expected Status Code}: \texttt{200 OK}

\noindent
\textbf{Example Response}:
\begin{minted}{json}
{ 
  "error": "",
  "content": [
    {
      "id": "00000000-0000-0000-0000-000000000000",
      "t": "2018-01-01T12:00:00+00:00",
      "counts": [
        {
          "species_id": 12,
          "name": "Blackbird",
          "count": 1
        }
      ]
    }
  ]
}
\end{minted}


\subsection{Add one or more readings to a sensor pair}

\noindent
\textbf{Method}: \texttt{POST}

\noindent
\textbf{Request Format}: \texttt{JSON}
\begin{minted}{json}
[
  { 
    "t": "2018-01-01T12:00:00+00:00",
    "counts": [
      {
        "species_id": 12,
        "count": 1
      }
    ]
  }
]
\end{minted}

\noindent
\textbf{Required Fields}: \texttt{t, counts}

\noindent
\textbf{Response Content-Type}: \texttt{JSON}

\noindent
\textbf{Expected Status Code}: \texttt{201 Created}

\noindent
\textbf{Example Response}:
\begin{minted}{json}
{ 
  "error": "",
  "content": [
    {
      "id": "00000000-0000-0000-0000-000000000000",
      "t": "2018-01-01T12:00:00+00:00",
      "counts": [
        {
          "species_id": 12,
          "name": "Blackbird",
          "count": 1
        }
      ]
    }
  ]
}
\end{minted}