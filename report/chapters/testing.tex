\chapter{Testing}

\section{\acrshort{api} Server Testing}

To ensure that the \acrshort{api} server works with as few errors and
incorrect responses as possible, a series of automated test suites were set
up that could be run on the codebase. Two different types of test were
used\textemdash{}functional (unit) tests, that run the actual models and
methods to ensure they act as expected, and integration tests, that perform
actual \acrshort{http} requests to the server to check that it returns the
expected output, and that the output is formatted correctly.

\subsection{Testing Framework}
The Jest~\cite{jest} framework, an open source testing framework maintained
by Facebook, was the natural choice of test framework for the project, thanks
to its ease of setup and use, as well as a very straightforward library. Jest
exposes an \texttt{expect} object that allows the developer to write
\acrfull{bdd}-style tests that read almost like sentences, such as ``expect
this status code to be 200''. The Jest project also features very extensive
documentation about using the \texttt{expect} object on their
website~\cite{jest-expect}.

To ensure strict code style and readability, a code linter was also
integrated into the project. The chosen linter was \texttt{ESLint}, a popular
JavaScript linter that can highlight formatting issues, inaccessible
(``dead'') code, and other style violations that may lead to unintentional
behaviour. One example of this is comparing two variables with the
\textit{equality} operator (\texttt{==}) rather than using the \textit{strict
equality} operator (\texttt{===}), which additionally tests for type equality
as well as value equality.

Formatting rules specified in the Airbnb JavaScript Style
Guide~\cite{airbnb-javascript} were used as the main rules for the linter, as
well as rules introduced by the \texttt{Prettier} formatter, which formats
the code when the user saves. The Airbnb JavaScript Style Guide is a popular
code style for JavaScript\textemdash{}as of 2018-03-23, nearly seventy
thousand users have ``starred'' the GitHub
repository~\cite{airbnb-javascript}. This means that it is a rather standard
way of formatting code, and should help to increase the chance of any
possible collaboration in the future.

A test script was also added to the root of the project, inside the
\texttt{package.json} file, that allows for all of the project's tests to be
run at once. This was then integrated with the JavaScript packages
\texttt{husky} and \texttt{lint-staged} so that the code was linted,
formatted, and all the tests run prior to a Git commit being created. The
commit would then be cancelled if there was a linter error or one or more
tests failed.

\subsection{Creating The Unit Tests}
The unit tests were derived from the expectations held on how the code
representations of the data models (for instance, the JavaScript classes that
represent the base stations, sensor pairs and readings) are initialised with
correct values, and how they validate their possible input values correctly.
For instance, each sensor pair has a \texttt{camera\_id} and a
\texttt{motion\_id} that stores the identifiers for the camera and motion
sensors in that pair. Those IDs should be positive integers, so tests must be
devised that ensure that the model accepts positive integers and rejects any
other input, by testing the model against as many different inputs as
possible.

\subsection{Creating The Integration Tests}

The integration tests involve testing that the server as a whole works as
intended; that when an end user sends correctly-formatted input they get the
expected output, and if they do not, then they get the correct error message.
This is sometimes referred to as \textit{black-box testing}, since you are
not testing the code itself\textemdash{}rather, you are testing the
\textit{system} as if you are the end user.

To help set up and run the \acrshort{http} requests and test the server
responses, the \texttt{supertest} library~\cite{supertest} was used. However,
one component of the system that isn't present during testing is the
database. Fortunately, a library exists~\cite{mock-knex} that can ``mock''
(replicate the functionality) of the database connection library so that
response to database queries coming from the server can be crafted to match
various scenarios, such as no matching data existing on the database, or
mocking the data being written to the database.

\subsection{Test Results And Coverage Report}

In the submitted \acrshort{api} code, all forty-six of the tests written pass
successfully, and none are failing, which is what is expected from code that
was written using the \acrfull{tdd} methodology (write the tests, then write
the code to pass the tests). This also reflects how code would be written in
industry to meet quality control standards. However, the test pass rate is
only one half of the story, because it counts for nothing if the tests don't
cover enough of the application's code. Jest provides a test coverage
analyser as part of its test suite, which analyses the project's code and
returns a percentage of how much of the code is covered by tests.
Table~\ref{tab:testcoverage} shows the output from the Jest test coverage
report. Overall, the tests cover 82.9\% percent of all code statements made,
60.7\% of all possible code branches, 73.1\% of all functions, and 82.7\% of
all lines.

This statistic is reasonably high for a small project, but to improve
reliability of the server, more tests can and should be added in the future.
Ideally, one should always aim to achieve 100\% test coverage, but time
constraints can impair this. Zhu et al discuss the various merits of these
test coverage metrics in great detail in \textit{Software Unit Test Coverage
and Adequacy}~\cite{zhu1997software}, but combined these metrics serve as a
useful insight into how effectively the tests actually test the code written.

All of the server tests can be found in the source code. Any file named
\texttt{*.test.js} will contain tests.

\begin{table}[]
  \centering
  \label{tab:testcoverage}
  \begin{tabular}{|l|lllll|}
    \hline
  File           & \% Stmts & \% Branch & \% Funcs & \% Lines & Uncovered Line \#s \\ \hline
  \textbf{All files}      & \textbf{82.91}    & \textbf{60.71}     & \textbf{73.17}    & \textbf{82.74}    &                    \\
  srv            & 76.3     & 52.27     & 55       & 76.3     &                    \\
  baseStation.js & 94.64    & 66.67     & 100      & 94.64    & 46,71,108          \\
  index.js       & 89.19    & 61.11     & 60       & 89.19    & 30,58,62,63        \\
  knexfile.js    & 100      & 100       & 100      & 100      &                    \\
  sensorPair.js  & 39.02    & 0         & 0        & 39.02    & ... 69,70,71,73,77 \\
  srv/models     & 96.55    & 91.67     & 90.48    & 96.43    &                    \\
  BaseStation.js & 100      & 100       & 100      & 100      &                    \\
  Reading.js     & 96.43    & 83.33     & 88.89    & 96.15    & 115                \\
  SensorPair.js  & 93.75    & 100       & 83.33    & 93.75    & 57                 \\
  index.js       & 100      & 100       & 100      & 100      &                    \\
  srv/test       & 100      & 100       & 100      & 100      &                    \\
  db.js          & 100      & 100       & 100      & 100      &                   \\ \hline
  \end{tabular}
  \caption{Jest test coverage report, generated on Monday 23rd April 2018.}
\end{table}