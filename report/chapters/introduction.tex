
\chapter{Introduction and background}

\section{Motivation and need}
Wildlife conservation experts constantly need to keep track of the location
and movements of wildlife for a number of reasons, including monitoring
species migration patterns and population
counts~\cite{karanth1995estimating}. Other options exist, like line transect
surveys (counting animals or traces of animals, like tracks and droppings) or
track surveys (physically visiting the area and counting animals); but in the
comparison study by Silveira et al~\parencite{silveira2003camera}, they found
that camera traps, despite their longer initial setup time and cost, ``can be
handled more easily and with relatively [lower] costs in a long term run''.

Based on this, it would be logical to assume that running costs and analysis
time could be reduced further by automating the classification of photos
taken by camera traps and sending the results back to a web server. This
would enable research teams to store, access, analyse and visualise wildlife
counts with ease, using an \acrfull{api} provided by the web service.

The biggest advantage of automating the classification stage of camera trap
studies is that it would save a lot of time after the main study has ended,
and results can be analysed as soon as possible.

\section{Requirements}

A list of requirements for the solution were reasonably easy to devise. Most
of the requirements are defined by the limits of the environments where this
system may be deployed, such as woodlands, grasslands, and national parks.

Most of the locations where this solution could be deployed may have very
limited cell network coverage, and definitely would not have WiFi
connectivity available. Therefore, the system would have to use
\gls{lorawan}, which has a theoretical range of up to twenty kilometres.
However, the lower data rate of \gls{lorawan} means that photos captured by
the camera traps would not be able to be sent back to a web server, so any
kind of image processing and classification would have to be performed
on-device.

Another limitation is the devices being used for the project. The main ``base
station'' device is the Creator Ci40 developer board, designed to ``allow
developers to rapidly create connected products''~\cite{creatorci40}. This
ability to rapidly prototype on the board, which has a \acrshort{mips}-based
processor and runs the OpenWRT Linux distribution. It also contains a WiFi
radio, useful for communicating with the device and debugging code on it
during development, and a \gls{6lowpan} radio, useful for communicating with
nearby devices.

The sensor devices were also provided as part of the project. They consist of
each sensor board integrated onto a \textit{MikroElektronika} \gls{6lowpan}
clicker board~\cite{mikroeclick}. This board runs a \acrfull{rtos}, which
allows the board to respond very quickly to changes in sensor input, as well
as the ability to be battery powered and the inclusion of a \gls{6lowpan}
radio, to communicate with the base station.

Another requirement arising from the intended deployment scenario is that the
system needs to be able to run on battery power, or indeed a low-voltage
power source, for a considerable amount of time. The intention of the project
is to reduce long-term study costs, and a need to replace sensor batteries
regularly would be failing this. Consequently, the system would have to rely
on hardware interrupts as much as possible, so that the devices can remain in
a ``sleep mode'' when they are not needed.

\section{Research and Literature Review}

The next step after gathering these requirements from the supervisor was to
find existing research and solutions to similar problems. These papers,
articles and reports have been compiled below.

\subsection{Similar Projects, Papers and Articles}

\subsubsection{Estimating tiger Panthera tigris populations from camera-trap
data using capture—recapture models}
