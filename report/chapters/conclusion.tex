\chapter{Evaluation and Conclusion}

\section{Evaluation of Expected Deliverables}
Below is a list of the expected outcomes and deliverables specified at the
start of the project and if they have been achieved (and to what extent).

\subsection{Base Station Program}
\textit{``A program built for the Creator Ci40 prototyping board that can
read in an image sent by the external camera module, and classify the types
and counts of animals in the image, before sending this data to a base
station via a LORA connection.''}

\noindent
This deliverable was only partially completed. A small program was created
that can handle sensors connecting to the base station, as well as receiving
motion events, but the classification program was not built.

\subsection{Motion Sensor Program}
\textit{``A program for the IR clicker device to detect movement and send a
message to a nearby camera device.''}

\noindent
This deliverable was completed and works to a reasonable standard, despite
issues with the onboard wireless radios.

\subsection{Camera Sensor Program}
\textit{``A program for the camera device to take a photo when it receives a
command via 6LoWPAN and send it to the Ci40 board.''}

\noindent
This deliverable was not completed, because of multiple issues using the
provided camera sensor and getting any readable data from it.

\subsection{Cloud Server}
\textit{``A simple cloud service to store and display results received from the
prototype boards''.}

\noindent
This deliverable was completed to a high standard.

\subsection{Tests}
\textit{``A rigorous testing regime for as much code as possible.''}

\noindent
This deliverable was met for the cloud server, as well as acceptance testing being conducted for the motion sensor.

\subsection{Real-Life Test}
\textit{``(If possible) a real-life test of the system in an uncontrolled
environment (i.e. a park or zoo).''}

\noindent
This was not met due to the other incomplete deliverables.

\section{Evaluation of Achievements}
Despite not all goals being met, the project should overall be considered
successful. The research conducted early in the project highlighted a clear
need for such an autonomous system to be developed, and recent strides in the
field of computer vision mean that such systems are becoming faster and
cheaper to deploy. However, it was a shame that not all of the deliverables
could be completed, especially being able to test the system in a real-life
setting.

Some of the biggest and most persistent issues throughout the project were
problems developing on the given hardware and using it to the best of its
ability. These problems were discussed in the \textit{Implementation}
chapter, including lack of documentation regarding the libraries used to
connect to sensor and hardware, as well as wireless connections not working
as intended. In hindsight, perhaps it would have been best to swap the
hardware out for better documented, more reliable (although perhaps a lot
less realistic) hardware, like the Arduino platform~\cite{arduino}. The
devices may not be as energy efficient as the ones used in this project, but
that could be considered a less important characteristic for a technical
demonstration.

Also, it may have been wiser to begin researching and developing the deep
learning classifier earlier on in the project. The work plan, devised at the
start of the project, mentioned researching and developing this section of
the project from November onwards, whereas perhaps research into different
techniques and frameworks should have been made earlier on.

\section{Possible Future Work}
There is a lot of promise in developing deep learning solutions to such a
problem as automatically classifing animals in camera trap images, and a lot
of recent breakthroughs have been extremely exciting\textemdash{}including
the work of Elias et al~\cite{elias2017s} whose work was published since the
start of this project. Perhaps the rise of general yet powerful models for
computer vision such as Inception-v3~\cite{szegedy2016rethinking} remove most
of the need for bespoke models to be created to solve these kinds of
problems, allowing more and more people to build exciting and powerful
applications for computer vision classification.

Additionally, leaps in the field of mobile computing are also incredibly
exciting. New development boards are appearing every day that can run on very
small amounts of power, vastly increasing expected lifetimes of such projects
as this one, as well as reducing costs needed to maintain and replace
sensors; perhaps future work in this domain could include building clustered
networks of sensors that can be highly resilient to link disruption, yet
still consume just a small amount of power.