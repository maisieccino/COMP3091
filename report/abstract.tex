Conservation experts, park rangers, and biologists frequently aim to try to
track the location and count of various species of animals. This is, more
often than not, extremely time consuming, since researchers have to install
camera traps with motion sensing shutters, and manually look back through
images to identify and count the animals.

This project and report explores the possibility of using a low-power
computer with sensors, connected to a web server over a wireless Internet
connection (a paradigm frequently referred to as \textit{The \acrfull{iot}})
to automate this task to save researchers hours of time when conducting
studies using camera traps.

The project also explores various methods in which species of animals could
be identified automatically, given various constraints of how the system can
work, including the availability and speed of the network link.
