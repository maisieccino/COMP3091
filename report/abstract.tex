Conservation experts, park rangers, and biologists frequently aim to try to
track the location and count of various species of animals for a number of
reasons, such as preventing illegal poaching and hunting, monitoring
biodiversity, and analysing migration patterns. This is, more often than not,
extremely time consuming, since researchers have to install camera traps with
motion sensing shutters, and manually look back through images to identify
and count the animals.

This project and report explores the possibility of using a low-power
computer with sensors, connected to a web server over a wireless Internet
connection (a paradigm frequently referred to as \textit{The \acrfull{iot}})
to automate this task to save researchers many hours of time when conducting
studies using camera traps.

The project also explores various methods in which species of animals could
be identified automatically using deep learning techniques, such as using
convolutional neural networks, with the constraint of having to operate on
remote, low-powered hardware with limited Internet connectivity.

The project proved to be a lot more difficult than first anticipated, thanks
to unforseen issues with the hardware being used, as well as difficulties in
developing a rigourous image classifier. However, it proved to be an
incredibly valuable learning experience, and provided the opportunity to work
in a problem domain that would not normally be possible in an undergraduate
computer science degree.