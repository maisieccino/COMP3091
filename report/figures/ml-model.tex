% thanks to https://github.com/jettan/tikz_cnn
%
%
\begin{figure}
	% \noindent\resizebox{\textwidth}{!}{
  \centering
	\begin{tikzpicture}
		\draw[use as bounding box, transparent] (-1.8,-1.8) rectangle (5.2, 3.2);

		% Define the macro.
		% 1st argument: Height and width of the layer rectangle slice.
		% 2nd argument: Depth of the layer slice
		% 3rd argument: X Offset --> use it to offset layers from previously drawn layers.
		% 4th argument: Options for filldraw.
		% 5th argument: Text to be placed below this layer.
		% 6th argument: Y Offset --> Use it when an output needs to be fed to multiple layers that are on the same X offset.

		\newcommand{\networkLayer}[6]{
			\def\a{#1} % Used to distinguish input resolution for current layer.
			\def\b{0.02}
			\def\c{#2} % Width of the cube to distinguish number of input channels for current layer.
			\def\t{#3} % X offset for current layer.
			\ifthenelse {\equal{#6} {}} {\def\y{0}} {\def\y{#6}} % Y offset for current layer.

			% Draw the layer body.
			\draw[line width=0.25mm](\c+\t,0,0) -- (\c+\t,\a,0) -- (\t,\a,0);                                                      % back plane
			\draw[line width=0.25mm](\t,0,\a) -- (\c+\t,0,\a) node[midway,below] {#5} -- (\c+\t,\a,\a) -- (\t,\a,\a) -- (\t,0,\a); % front plane
			\draw[line width=0.25mm](\c+\t,0,0) -- (\c+\t,0,\a);
			\draw[line width=0.25mm](\c+\t,\a,0) -- (\c+\t,\a,\a);
			\draw[line width=0.25mm](\t,\a,0) -- (\t,\a,\a);

			% Recolor visible surfaces
			\filldraw[#4] (\t+\b,\b,\a) -- (\c+\t-\b,\b,\a) -- (\c+\t-\b,\a-\b,\a) -- (\t+\b,\a-\b,\a) -- (\t+\b,\b,\a); % front plane
			\filldraw[#4] (\t+\b,\a,\a-\b) -- (\c+\t-\b,\a,\a-\b) -- (\c+\t-\b,\a,\b) -- (\t+\b,\a,\b);

			% Colored slice.
			\ifthenelse {\equal{#4} {}}
			{} % Do not draw colored slice if #4 is blank.
			{\filldraw[#4] (\c+\t,\b,\a-\b) -- (\c+\t,\b,\b) -- (\c+\t,\a-\b,\b) -- (\c+\t,\a-\b,\a-\b);} % Else, draw a colored slice.
		}

		\networkLayer{3.0}{0.03}{-0.5}{color=gray!80}{input}

		\networkLayer{2.5}{0.1}{0.0}{color=white}{conv}{}    % S1
    \networkLayer{2.5}{0.1}{0.2}{color=white}{}{}        % S2
    
    \networkLayer{1.5}{1.5}{0.5}{color=white}{pooling}{}

		% OUTPUT
		\networkLayer{0.5}{3.0}{2.0}{color=red!40}{output}{}          % Pixelwise segmentation with classes.


	\end{tikzpicture}
	% }
 \caption{A visualisation of how the convolutional neural network would have
 been structured, with each image inputted as a square image with red, green,
 and blue colour channels; the one or more convolutional layers which filter
 for local features; a pooling layer which groups together low-level features
 to find high-level image patterns; and finally an output layer which ouputs
 a one-dimensional list of probabilities for each animal species.}
	\label{fig:cnn}
\end{figure}